\documentclass[a4paper, 12pt]{article}

\usepackage{enumerate}
\usepackage{hyperref}
\hypersetup{
	colorlinks=true,
	linkcolor=blue,
	filecolor=blue,
	urlcolor=blue,
	citecolor=blue,
}
\usepackage{amsmath}
\usepackage{amsthm}
\usepackage{amssymb}
\usepackage[margin=3cm]{geometry}
\usepackage{mathpazo}
\usepackage{url}
\usepackage[labelformat=simple]{subcaption}
\usepackage{tikz}
\usepackage{pgf}
\usepackage{longtable}
\usepackage{multirow}
\usepackage{graphicx}

\begin{document}
\pagestyle{empty}

\begin{center}
{\Large Computer Science (CS102-2)} 

\vspace{0.25cm}

{\large Joshua Maglione}

\vspace{0.25cm}

Semester 2 (2025)
\end{center}

\vspace{0.5cm}

\begin{description}
    \item[Module information:] \hfill
    \begin{description}
      \item[Coordinates:] \hfill \\ [0.5em]
      % \begin{center}
        \begin{tabular}{rcl}
          Wednesdays & 2:00pm -- 2:50pm & \href{https://clients.mapsindoors.com/nuigalwayweb/9167eab0dc78437c93c76b57/details/22185f1db04149f3ab4bac48}{AC202} \\ Thursdays & 9:00am -- 9:50am & \href{https://clients.mapsindoors.com/nuigalwayweb/9167eab0dc78437c93c76b57/details/76ccb06aa36c4758823836e7}{Tyndall}
        \end{tabular}
      % \end{center}
      \item[Contact:] \hfill \\ \href{mailto:joshua.maglione@universityofgalway.ie}{\texttt{joshua.maglione@universityofgalway.ie}}
      \item[Website:] \href{https://universityofgalway.instructure.com/}{\textsf{Canvas}} and \url{https://joshmaglione.com/2025CS102-2.html} 
    \end{description} 
    \vspace{1cm}
    \item[Topics:] This is the ``main lecture'' of the second semester of CS102. This module is supplemented by CS102-4 Further Computing. We will cover the basics algorithms and data structures. We cover topics of
    \begin{enumerate} 
      \item computational complexity (big O notation),
      \item fundamental data structures,
      \item sorting algorithms,
      \item recursion.
    \end{enumerate} 
    We will be using Python throughout for examples.
    \vspace{1cm}
    \item[Labs:] This module also has two 3-hour labs each week. You are required to attend at least one of these labs per week. Each week you can independently choose either lab. Labs begin during the third week---the week of 27 January. There will be no labs during the fourth week---the week of 3 February (St. Brigid's Day).
    
    Labs will have a few demonstrators to help with the lab assignment. 
    \begin{description}
      \item[Coordinates:] \hfill \\[0.5em]
      \begin{tabular}{rcl}
        Mondays & 3:00pm -- 6:00pm & Block E G-013 (BLE-G013) \\ Wednesdays & 3:00pm -- 6:00pm & Block E G-013 (BLE-G013)
      \end{tabular}
    \end{description}
    \vspace{1cm}
    \item[Assessment:] There will be 7 lab assignments due on Fridays at 5:00pm. The final mark consists of $40\%$ assessment and $60\%$ final exam.
    \vspace{1cm}
    \item[Reading:] \hfill 
    \begin{enumerate}
      \item Bhargava, Aditya Y. \emph{Grokking Algorithms}. Simon and Schuster, 2024.
      
      First ed.\ available \href{https://edu.anarcho-copy.org/Algorithm/grokking-algorithms-illustrated-programmers-curious.pdf}{online}; second ed.\ available on publisher's \href{https://www.manning.com/books/grokking-algorithms}{website}.

      \item Cormen, Thomas H., et al. \emph{Introduction to Algorithms}, third edition. MIT press, 2022.
      
      Available through the University Library and \href{https://search.library.nuigalway.ie/permalink/f/1pmb9lf/353GAL_ALMA_DS2139494700003626}{online}.

      \item Lehman, Eric, et al. \emph{Mathematics for Computer Science}, version 6 June 2018. 
      
      Available \href{https://courses.csail.mit.edu/6.042/spring18/mcs.pdf}{online}.

      \item Morin, Pat. \emph{Open Data Structures (in Pseudocode)}, edition $0.1\text{G}\beta$.
      
      Available \href{https://opendatastructures.org/}{online}. 
    \end{enumerate}
\end{description}



\end{document}
